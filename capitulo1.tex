\chapter{Contextualización}

\section{Antecedentes}
%Referencia a todos los trabajos (\emph{Recientes, < 4 años})
%Quien ha trabajado en ese tema. ¿Qué hizo? ¿Cómo lo hizo?
%¿Qué métodos utilizó?
%Mismo estilo para números y títulos

\section{Justificación}


\section{Objetivos}
%En este trabajo nos hemos planteado el siguiente objetivo general:

\subsection{Objetivo general}
%Todo objetivo debe empezar con un verbo en su forma infinitiva,
%ejemplo: estudiar, definir, desarrollar, analizar, implementar
%Cada objetivo, general o específico debe representar algo
%específico sin dejar posibilidad a ambigüedades.
%Para cumplir con el objetivo planteado se require cumplir con los
%siguientes objetivos específicos:
\subsection{Objetivos específicos}
%\begin{itemize}
%\item Cada objetivo específico representa una actividad o tarea
%general del trabajo que se va a realizar. 
%\end{itemize}

\section{Metodología}

\subsection{Fase de diagnóstico}
%\begin{itemize}
%\item
%\end{itemize}

\subsection{Fase de diseño}
%\begin{itemize}
%\item
%\end{itemize}

\subsection{Fase de implementación}
%\begin{itemize}
%\item
%\end{itemize}

\subsection{Fase de pruebas}
%\begin{itemize}
%\item
%\end{itemize}

\section{Alcance}

\section{Cronograma de evaluaciones}


\pagebreak

\section{Cronograma de actividades}

\pagebreak