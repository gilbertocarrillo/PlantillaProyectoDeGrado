\documentclass[12pt,oneside]{book}
\usepackage{ulamonog} %proyecto de grado
%\usepackage[ansinew]{inputenc} % escribir acentos
\usepackage[utf8]{inputenc}
%\usepackage[activeacute,spanish]{babel}
\usepackage{moreverb}
\usepackage[none]{hyphenat}
\usepackage{color}
\usepackage{hyperref}
%\usepackage[open, openlevel=1]{bookmark}
\usepackage[spanish]{babel}
\selectlanguage{spanish}
\usepackage[utf8]{inputenc}
\usepackage[numbers,sort]{natbib}
\setcounter{secnumdepth}{4}
\setcounter{tocdepth}{4}
% ***************************************************************** %
% En el siguiente comando se pueden modificar (para el archivo .pdf):
% Titulo, Autor, Palabras clave
% ***************************************************************** %

% si se va a imprimir,
% se incluye al final despues de citecolor=blue, el comando draft=true

\hypersetup{
pdftitle={ProyectoDeGrado},
pdfauthor={Autor},
pdfsubject={Proyecto de Grado}, % se deja igual
pdfkeywords={Palabras,claves,muchas,pocas}, 
pdfstartview=FitH,
citecolor=blue,
colorlinks=true,
linkcolor=black,
%hidelinks
%draft=true
}

% Si desea que no aparezca la lista de tablas o figuras descomente las siguientes lineas
\nolistoftables
\nolistoffigures

% ***************************************************************** %
% FIN DE
% Titulo, Autor, Palabras clave
% ***************************************************************** %

\sloppy

\begin{document}
\frontmatter

% ***************************************************************** %
% Si sabe la fecha de presentación del Proyecto de Grado
% con o sin comentar
% ***************************************************************** %

%\fechaentrega{Junio} % Si sabe cuando se presentó
%\presentaciondia{2 de Junio} % si conoce exactamente el día
%\presentacionlugar{Laboratorio 226 de EISULA} % si conoce exactamente el lugar donde se presentó

% ***************************************************************** %
% FIN DE
% Si sabe la fecha de presentación
% ***************************************************************** %

% ***************************************************************** %
% Si el Proyecto de Grado tiene mencion especial
% con o sin comentar
% ***************************************************************** %

%\mencionespecial{Este proyecto fue seleccionado como \textbf{mejor proyecto de grado} de la Escuela de
%Ingeniería de Sistemas, en el IC aniversario de la Facultad de Ingeniería.} % si tiene mención especial

% ***************************************************************** %
% FIN DE
% Si sabe la fecha de presentación
% con o sin comentar
% ***************************************************************** %

% ***************************************************************** %
% Portada y resumen
% ***************************************************************** %
% Si desea el logo en la parte de abajo
%\logoabajo

% En caso de que el año sea diferente al actual, quite el comentario a la siguiente linea
%\copyrightyear{2007}

% Al final cuando hayan presentado, sin comentar
% deberia ser el número de tesis presentada y la opción, IO por ejemplo
% Numero del proyecto
%\numproy{00IO}

% Puede cambiar el tipo de monografía, por defecto: Proyecto de Grado
%\tipomonografia{Informe Final de Proyecto de Grado}

% En caso de hacer propuesta de proyecto, quitele el comentario a la
% siguiente linea --- En este caso no aparece ni la hoja de presentacion
% ni la hoja de dedicatoria
\propuesta

% Si se utiliza este formato para hacer un informe técnico, descomente las siguientes líneas
%\informe %
%\proyecto{} % Por ejemplo, puede colocar el nombre del proyecto al que esta asociado el informe técnico
            % Ejemplos: \proyecto{Convenio ULA PDVSA}
            % \proyecto{Canalizaciones del río Chama. Asesoría Alcaldía Municipio Libertador}

% Datos del proyecto de grado
\title{}
\author{}
\cedula{}
\tutor{}
%\cotutor{Dra. Puede Haber}
%\primerjurado{Dr. Primer Profesor}
%\segundojurado{Prof. Segundo Profesor}
%\tercerjurado{Prof. Tercer Profesor}

% NO TOCAR si es Ingenieria de Sistemas
%\grado{Ingeniero Químico} % por defecto Ingeniero de Sistemas
%\tutorname{Asesor} % por defecto es Tutor

%\signaturepage ----- NO TOCAR

\resumen{}

\descriptores{}

%\cota{IXD A01.1}

% Si es control y automatizacion se comenta la siguiente linea
\opcion{Sistemas Computacionales}

% ***************************************************************** %
% FIN DE
% Portada y resumen
% ***************************************************************** %


% ***************************************************************** %
% Si tiene dedicatoria
% con o sin comentar
% ***************************************************************** %
\dedicatoria{No hay dedicatoria}

% ***************************************************************** %
% FIN DE
% Si tiene dedicatoria
% ***************************************************************** %

\beforepreface

% ***************************************************************** %
% Agradecimientos y primer capitulo (sin numeracion)
% ***************************************************************** %

%\prefacesection{Agradecimientos}

%En esta sección se agradece a las personas que contribuyeron a llevar a buen final este trabajo.

%%% Capitulo sin numero, antes de la pagina 1
\prefacesection{Introducción}
%Un párrafo de introducción al área de trabajo (4 líneas)
\afterpreface 
\pagestyle{fancyplain}
\renewcommand{\chaptermark}[1]{\markboth{#1}{\textsc{\footnotesize\thechapter\ #1}}}
\renewcommand{\sectionmark}[1]{\markright{\textsc{\footnotesize\thesection\ #1}}}
\lhead[\fancyplain{}{\textsc{\footnotesize\thepage}}]%
{\fancyplain{}{\rightmark}}
\rhead[\fancyplain{}{\leftmark}]%
{\fancyplain{}{\textsc{\footnotesize\thepage}}} \cfoot{}
\mainmatter
  
%Ejemplo de citas \citep{Stall:04}	
%\emph{Ejemplo de itálicas}
%\bf{Ejemplo de negritas}
% ***************************************************************** %
% FIN DE
% Agradecimientos y primer capitulo (sin numeracion)
% ***************************************************************** %

% ***************************************************************** %
% Cuerpo
% ***************************************************************** %

\chapter{Contextualización}

\section{Antecedentes}
%Referencia a todos los trabajos (\emph{Recientes, < 4 años})
%Quien ha trabajado en ese tema. ¿Qué hizo? ¿Cómo lo hizo?
%¿Qué métodos utilizó?
%Mismo estilo para números y títulos

\section{Justificación}


\section{Objetivos}
%En este trabajo nos hemos planteado el siguiente objetivo general:

\subsection{Objetivo general}
%Todo objetivo debe empezar con un verbo en su forma infinitiva,
%ejemplo: estudiar, definir, desarrollar, analizar, implementar
%Cada objetivo, general o específico debe representar algo
%específico sin dejar posibilidad a ambigüedades.
%Para cumplir con el objetivo planteado se require cumplir con los
%siguientes objetivos específicos:
\subsection{Objetivos específicos}
%\begin{itemize}
%\item Cada objetivo específico representa una actividad o tarea
%general del trabajo que se va a realizar. 
%\end{itemize}

\section{Metodología}

\subsection{Fase de diagnóstico}
%\begin{itemize}
%\item
%\end{itemize}

\subsection{Fase de diseño}
%\begin{itemize}
%\item
%\end{itemize}

\subsection{Fase de implementación}
%\begin{itemize}
%\item
%\end{itemize}

\subsection{Fase de pruebas}
%\begin{itemize}
%\item
%\end{itemize}

\section{Alcance}

\section{Cronograma de evaluaciones}


\pagebreak

\section{Cronograma de actividades}

\pagebreak
%\chapter{Marco teorico}
ahasdflasdlfa
\section{Antecedentes}
%Referencia a todos los trabajos (\emph{Recientes, < 4 años})
%Quien ha trabajado en ese tema. ¿Qué hizo? ¿Cómo lo hizo?
%¿Qué métodos utilizó?
%Mismo estilo para números y títulos
En la investigación \cite{usuga2014diseno} titulada “Diseño de una unidad didáctica para la enseñanza-aprendizaje de la multiplicación de números naturales en el grado tercero de la Institución Educativa Antonio Derka Santo Domingo del municipio de Medellín”, proponen una unidad didáctica donde se plantean varias estrategias basadas en la lúdica para aprender las tablas de multiplicación. Dentro de la unidad didáctica se encuentran juegos como “el saca piojos multiplicativo”, el cual está basado en origami, cuyas desventajas son el no poder representar todas las tablas de multiplicación y no ser apto para todas las edades debido a las habilidades de origami requeridas para su construcción; el “rompecabezas multiplicativo” que consiste en u.
%\chapter{Marco teorico}
ahasdflasdlfa
\section{Antecedentes}
%Referencia a todos los trabajos (\emph{Recientes, < 4 años})
%Quien ha trabajado en ese tema. ¿Qué hizo? ¿Cómo lo hizo?
%¿Qué métodos utilizó?
%Mismo estilo para números y títulos
En la investigación \cite{usuga2014diseno} titulada “Diseño de una unidad didáctica para la enseñanza-aprendizaje de la multiplicación de números naturales en el grado tercero de la Institución Educativa Antonio Derka Santo Domingo del municipio de Medellín”, proponen una unidad didáctica donde se plantean varias estrategias basadas en la lúdica para aprender las tablas de multiplicación. Dentro de la unidad didáctica se encuentran juegos como “el saca piojos multiplicativo”, el cual está basado en origami, cuyas desventajas son el no poder representar todas las tablas de multiplicación y no ser apto para todas las edades debido a las habilidades de origami requeridas para su construcción; el “rompecabezas multiplicativo” que consiste en u.
%\chapter{Marco teorico}
ahasdflasdlfa
\section{Antecedentes}
%Referencia a todos los trabajos (\emph{Recientes, < 4 años})
%Quien ha trabajado en ese tema. ¿Qué hizo? ¿Cómo lo hizo?
%¿Qué métodos utilizó?
%Mismo estilo para números y títulos
En la investigación \cite{usuga2014diseno} titulada “Diseño de una unidad didáctica para la enseñanza-aprendizaje de la multiplicación de números naturales en el grado tercero de la Institución Educativa Antonio Derka Santo Domingo del municipio de Medellín”, proponen una unidad didáctica donde se plantean varias estrategias basadas en la lúdica para aprender las tablas de multiplicación. Dentro de la unidad didáctica se encuentran juegos como “el saca piojos multiplicativo”, el cual está basado en origami, cuyas desventajas son el no poder representar todas las tablas de multiplicación y no ser apto para todas las edades debido a las habilidades de origami requeridas para su construcción; el “rompecabezas multiplicativo” que consiste en u.
%\chapter{Marco teorico}
ahasdflasdlfa
\section{Antecedentes}
%Referencia a todos los trabajos (\emph{Recientes, < 4 años})
%Quien ha trabajado en ese tema. ¿Qué hizo? ¿Cómo lo hizo?
%¿Qué métodos utilizó?
%Mismo estilo para números y títulos
En la investigación \cite{usuga2014diseno} titulada “Diseño de una unidad didáctica para la enseñanza-aprendizaje de la multiplicación de números naturales en el grado tercero de la Institución Educativa Antonio Derka Santo Domingo del municipio de Medellín”, proponen una unidad didáctica donde se plantean varias estrategias basadas en la lúdica para aprender las tablas de multiplicación. Dentro de la unidad didáctica se encuentran juegos como “el saca piojos multiplicativo”, el cual está basado en origami, cuyas desventajas son el no poder representar todas las tablas de multiplicación y no ser apto para todas las edades debido a las habilidades de origami requeridas para su construcción; el “rompecabezas multiplicativo” que consiste en u.

%\bibliographystyle{apalikesp}
\bibliographystyle{IEEEannot}

% ***************************************************************** %
% Para agregar toda la bibliografia del archivo .bib
% solo descomente el siguiente comando
%\nocite{*}
% ***************************************************************** %
% ***************************************************************** %
\bibliography{bibliografia}
% ***************************************************************** %
% FIN DE
% Cuerpo
% ***************************************************************** %
\appendix
\scriptsize
%\verbatimtabinput[4]{pi.h}
%\subsection{sch\_pi.c}
%\verbatimtabinput[4]{sch\_pi.c}

\end{document}