\chapter{Marco teorico}
ahasdflasdlfa
\section{Antecedentes}
%Referencia a todos los trabajos (\emph{Recientes, < 4 años})
%Quien ha trabajado en ese tema. ¿Qué hizo? ¿Cómo lo hizo?
%¿Qué métodos utilizó?
%Mismo estilo para números y títulos
En la investigación \cite{usuga2014diseno} titulada “Diseño de una unidad didáctica para la enseñanza-aprendizaje de la multiplicación de números naturales en el grado tercero de la Institución Educativa Antonio Derka Santo Domingo del municipio de Medellín”, proponen una unidad didáctica donde se plantean varias estrategias basadas en la lúdica para aprender las tablas de multiplicación. Dentro de la unidad didáctica se encuentran juegos como “el saca piojos multiplicativo”, el cual está basado en origami, cuyas desventajas son el no poder representar todas las tablas de multiplicación y no ser apto para todas las edades debido a las habilidades de origami requeridas para su construcción; el “rompecabezas multiplicativo” que consiste en u.